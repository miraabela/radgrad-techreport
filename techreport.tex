\documentclass[english]{proposalnsf}
\usepackage{graphicx}
\usepackage{url}
\usepackage{hyperref}
\usepackage[square,numbers]{natbib}
\usepackage[nottoc,numbib]{tocbibind}

\title{Integrating a Recommendation System for Scholarships \& Internships along with Prioritized Opportunities}
\author{Mirabela Medallon \\ Information and Computer Sciences \\ University of Hawaii}

\begin{document}
	\maketitle
	\tableofcontents
	\newpage
	
	\section{Problem Statement}
	\label{introduction}
	
	Please view the following \href{https://drive.mindmup.com/map/1B_krIFKYAHRDOuSgDn2jVU5e9uZPPjw_}{MindMap} for a better overview 
	
	\subsection{Priority of Opportunities}
	When looking for events and opportunities, a student should be aware of the most urgent opportunity first, which is one with a close deadline or an event that only happens once. This allows them to efficiently make use of their time when searching and applying for opportunities. This prevents them from spending time combing through huge lists while the deadline of a opportunity of big importance to them passes, because they don't see it in time. 
	
	Currently, in the ICS Department, we are made aware of upcoming events usually through our department advisor's emails,  the Discord group, or with flyers. Our current advisor is retiring soon, and he won't be forwarding all the valuable emails and events anymore. Also, sometimes important announcements may get lost in the sea of emails we receive. Additionally, some email forwards that go to the whole ICS Undergraduate mailing list are not applicable to everyone. 
	
	In the current RadGrad system, the Explorer doesn't indicate which opportunities are currently open for applications and doesn't show the deadlines. 
	
	
	Solution: RadGrad users fill out profile. Save the future advisor time. Have them forward emails directly to RadGrad, which we will handle and post in a central system for students to search through. This allows them to see everything upcoming in one place, instead of searching through emails. Only relevant announcements will be prioritized
	
	\subsection{Scholarships}
	
	Many scholarship sites have lists but they are unstructured and take very long to read through. As a student, I just want to know a few key things about the scholarship to decide whether or not to invest time to apply. Most websites only highlight the title, amount, and a paragraph of the description. Details such as deadlines, demographic requirements (age, gender, veteran, disabled, legacy, race, location, etc), application requirements (transcripts, essays, letters of recommendation, etc) are not easily highlighted. Reading through every single scholarship description website is very time consuming, and I only am searching for a few details.
	
	
	\subsection{Internships}
	

	
	\section{High Level Goals}
	\label{Goals}
	\begin{itemize}
		\item  Better matching for employers, students fill out profile, we recommend students for opportunity
		\item Increase RadGrad interaction \& engagement
		\item Central place for ICS department information
		\item Track more of student's undergrad activity, better for resume generation
		\item Minimize Advisor workload
		\item  Increased connection between interactions (classes, opportunities, internship/scholarship search, announcements/ notifications)
		\item Easier search for students
	\end{itemize}
	
	
	
	
	\section{Proposed Solution Design}
	\label{solution-design}
	
		\subsection{Web Scraping}
		For scholarships, I plan to scrape the web for specifically Computer Science scholarships and automatically extract details and fill out fields which will create an easily searchable and filterable database of scholarships. I plan to use Topic Modeling to summarize descriptions to reduce reading and search time for relevant scholarships. The following information will be extracted for each scholarship:
		
		\textbf{Demographic Requirements}
		\begin{itemize}
			\item age, gender, physical location
			\item diversity: veteran, disabled, race
			\item legacy student
			\item major, career interest
		\end{itemize}
	
			\textbf{Application Requirements}
		\begin{itemize}
			\item transcripts, essays, letters of recommendation
		\end{itemize}
		
		\textbf{Information}
		\begin{itemize}
			\item award amount
			\item title, desription, link
			\item deadline (HST)
			\item contact information
		\end{itemize}
		
		
		For internships in addition to regular referrals from advisors (see section below) I will supplement the internship database with scraped data from job board sites. The following information will be extracted for each internship:
		
		\textbf{Demographic Requirements}
		\begin{itemize}
			\item title, desription, link
			\item paid? if so, rate
			\item link, contact information
		\end{itemize}
		
		\textbf{Application Requirements}
		\begin{itemize}
			\item skills (languages, technologies) required
			\subitem this will match to the student profile Interests
			\item transcripts, essays, letters of recommendation
		\end{itemize}
	
		\textbf{Information}
		\begin{itemize}
			\item age, gender, physical location
			\item diversity: veteran, disabled, race
			\item graduation date
			\item major, career interest
		\end{itemize}
		
		
		\subsection{Email Forwards to Opportunities}
		
		
		\subsection{Recommendation and Profile Matching}
		Each student should fill out their profile to the fullest to ensure the best recommendations. Their profile will consist of their Major, Careers, Interests (see below how it can be translated to internship skills), and Demographic Information. 
		
		For internships, students with similar Interests, Opportunities, or Classes as the internship Skills requirements will have a recommendation score.
		
		Verifying Skills:
		Some students may just list Interests as a curiosity interest, not something they actually know yet, so when a student selects an interest, an additional question can be selected "Do you know this already?" " KNOWN SKILL vs INTEREST"
		
		
		
		\subsection{Dates, Calendars, and a Priority System}
		
	
\end{document}

