\documentclass[english]{proposalnsf}
\usepackage{graphicx}
\usepackage{url}
\usepackage{hyperref}
\usepackage[square,numbers]{natbib}
\usepackage[nottoc,numbib]{tocbibind}
\usepackage{outlines}
\usepackage{wasysym}

\title{"Skills" for Career Guidance and Discovery of Relevant Experiences}
\author{Mirabela Medallon \\ Information and Computer Sciences \\ University of Hawaii}

\begin{document}
	\maketitle
	\tableofcontents
	\newpage
	
	\section{Introduction}
	\label{introduction}


	\subsection{Research Questions}
	\label{questions}

My project aims to answer the following questions: 
\begin{outline}
    \1 If RadGrad was a central place for internship search, local job search, and career help, will it improve engagement and an overall view of RadGrad?
    \1 How can RadGrad speed up the internship and local job search process?
    \1 Will RadGrad see an increase in interaction with other parts of the app (Opportunity, Careers, and Classes) if an additional Jobs component was added that encourages further exploration?
\end{outline}

	\section{Related Work}
	\label{relatedwork}


	\section{Research Design}
	\label{design}
	My project proposes: "If a central job and internship recommendation and search system was created for ICS Students to find local and mainland opportunities, then the students will have a higher success rate with searching and applying for internships." In order to understand the current problems the students face, and where each problem comes from, I propose to ask the following questions:
	
	About sources to find opportunities:
\begin{outline}
    \1 What sources do you use to find internship, research, or professional experience related to your major? 
        \2[\Square] Google
        \2[\Square] Online Job Boards
        \2[\Square] SECE
        \2[\Square] Career Fairs and Conferences
        \2[\Square] Advisor
        \2[\Square] Manoa Career Center
        \2[\Square] ACM Manoa
        \2[\Square] RadGrad
        \2[\Square] Word of Mouth
        \2[\Square] Networking & Personal Connections
    \1 Of the sources that you selected, rate the effectiveness of each:
        \2 1-5 For each
    \1 Which source(s) resulted in getting an interview offer or acceptance?
\end{outline}

	About the important, relevance, & qualifications:
\begin{outline}
    \1 How important do you think it is to get an internship, research, or professional experience related to your major before graduation?
        \2[\Square] Not important
        \2[\Square] Slightly Important
        \2[\Square] Mostly Important
        \2[\Square] Very Important
    \1 Did you or will you make any effort to get an internship, research, or professional experience related to your major before graduation?
        \2[\Square] I already did, will try again
        \2[\Square] I already did, will not try again
        \2[\Square] I didn’t yet, but plan to
        \2[\Square] I didn’t yet, and don’t plan to
    \1 If you did make an effort, how many opportunities did you FIND that were interesting and relevant to you? (Regardless of whether or not you were accepted or applied)
        \2[\Square] None
        \2[\Square] Some
        \2[\Square] Mostly
        \2[\Square] All
    \1 Of those opportunities, how qualified did you feel based on the requirements in the description?
        \2[\Square] 1 - Unqualified
        \2[\Square] 2 - Slightly Qualified
        \2[\Square] 3 - Moderately Qualified
        \2[\Square] 4 - Mostly Qualified
        \2[\Square] 5 - Extremely Qualified
    \1 When you read the requirements description of the postings, how many skills and technologies were unfamiliar to you or you felt you lacked in?
        \2[\Square] None
        \2[\Square] A Little
        \2[\Square] Some
        \2[\Square] Most
        \2[\Square] All
    \1 Of the technologies that were unfamiliar to you, did you or will you try to learn more about them?
        \2[\Square] I already did, will try again
        \2[\Square] I already did, will not try again
        \2[\Square] I didn’t yet, but plan to
        \2[\Square] I didn’t yet, and don’t plan to
    \1 Where did you go to learn more about them?
        \2[\Square] Google
        \2[\Square] Asked Advisor / Professor
        \2[\Square] Took a class
        \2[\Square] Bookstore
\end{outline}

About the search process:
\begin{outline}
    \1 Do you think the searching for internships is difficult?
        \2[\Square] Not at all
        \2[\Square] Somewhat
        \2[\Square] Mostly
        \2[\Square] Very
    \1 If so, which part of the internship search process is difficult?
    \1 How many internships did you apply to, if at all?
        \2[\Square] None
        \2[\Square] 1-5
        \2[\Square] 5-10
        \2[\Square] 10-20
        \2[\Square] 20+
    \1 Of the ones you applied to, how many did they proceed with an interview for?
        \2[\Square] 75-100%
        \2[\Square] 50-75%
        \2[\Square] 25-50%
        \2[\Square] Less than 25%
        \2[\Square] None
    \1 Of the ones you interviewed for, how many gave you an offer?
        \2[\Square] 75-100%
        \2[\Square] 50-75%
        \2[\Square] 25-50%
        \2[\Square] Less than 25%
        \2[\Square] None
\end{outline}

About local vs mainland opportunities:
\begin{outline}
    \1 Is it easier to find local internships or mainland internships?
    \1 Do you think you have higher chances of landing a local or mainland internship?
    \1 Do you know a source to find a lot of local opportunities?
    \1 If you were more aware of local opportunities, would you apply?
\end{outline}

	\section{High Level Goals}
	\label{Goals}
	I plan to design a system that will achieve the following goals: increase overall RadGrad engagement, become a central and one of the first places to turn to when a student needs to look for internships, aid advisor workload, and quicken the search process for a matching internship. 

	
	\section{System Overview}
	\label{overview}

The system will be a central database that aggregates data from multiple sources to make the searching process more streamlined and simpler. It will do this by scraping SECE, various job board websites, and creating a system for local employers to email RadGrad instead of the advisor or ACM Manoa. Alternatively, there can be a "portal" (or a Google Form) for local employers to request to post jobs for our students only. Ideally, RadGrad is the first source that 3rd parties should contact to announce and notify students of their opportunities. 

In order to have better recommendations and match students with their previous experience, the list of ICS Classes that is imported will additionally be added "Skill" values that the student can edit and update. Since student opportunities also count as previous experience, the opportunities that a student has done will also be added "Skill" values. The student could have a "Profile" page that allows them to update their skills, interests, career goals. This page would allow them to view how different experiences (classes \& opportunities) contributed to the skills they know. \textit{This profile page could also show them the progress they are at to fulfilling all the skills required for one of their career goals they selected.}

Ideally, the system would be embedded into RadGrad that operates independently as a 3rd party service. Requests sent to the system will include information about the student's profile, and a response will give the list of matching opportunities. 

One of my thoughts is that students may not be using RadGrad that much because they might feel it is not updated often. For example, they think it is something to check every semester of so. Another reason they might turn to other sources is because it is updated often, such as the ACM Manoa Discord and their Inbox. This can be combated with this new system that regularly updates and keeps track of deadlines. 

Since students must verify their opportunities manually by email before it is included in their profile, I feel this might hinder the speed at which new "Skills" gained from these opportunities can be added to a students profile. So, a student's matches might not be completely accurate because they did some opportunities but are still waiting for verification and for those skills to be added. 

Instead of receiving multiple emails a day, or needing to check different sources frequently, students can just subscribe to a weekly email that summarizes the most relevant opportunities for them.

	
	\section{Evaluation}
	\label{evaluation}
	Some metrics that can be compared to evaluate the effectiveness of my system are: the number of logins, click through rate of emails, and page view counts. An additional survey can be conducted hear personal responses and comments about the system.
	
	
\end{document}

